% Resumo em l�ngua estrangeira (em ingl�s Abstract, em espanhol Resumen, em franc�s R�sum�)
\begin{center}
	{\Large{\textbf{BNativus: free platform to practice the study in foreign language}}}
\end{center}

\vspace{1cm}

\begin{flushright}
	Author: Jonas Elan Teixeira Alves\\
	Supervisor: Prof. Mestre Leonardo Reis Lucena
\end{flushright}

\vspace{1cm}

\begin{center}
	\Large{\textsc{\textbf{Abstract}}}
\end{center}

\noindent Currently in Brazil, having mastery over a foreign language, especially English, is no longer a differential and has become an obligation for those who seek to advance professionally in technological, research or even personal interests. Achieving this ability enables new opportunities to be created, such as world travel, better job offers, student exchange, and more. There are several ways to learn a second language that have already been consolidated, but the most effective ones are sometimes expensive for the low-income population. Alternatives can be found on the WEB, but there is no place where these people, who are studying the foreign language, can interact, requiring students to look for others with the same interest in the forums and other needs, personal, such as email or number, may be a problem, as this data will be available to other users. In this scenario, a free and open environment for the community was designed for students of new languages that engage in others, building a community that shares knowledge. To validate an idea, a search was conducted through a form and conversations, with the purpose of shaping a way to solve this problem. As a result, a web system was developed, with the minimum of planned features, with the intention of being consolidated nationally and later differently in other places in the world. Covering interaction via video and text, its development phase was carried out at the IFRN.

\noindent\textit{Keywords}: Foreign language, Study, Web plataform , Community.