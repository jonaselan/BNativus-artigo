% Resumo
\begin{center}
	{\Large{\textbf{BNativus: plataforma gratuita para praticar o estudo em uma língua estrangeira}}}
\end{center}

\vspace{1cm}

\begin{flushright}
	Autor: Jonas Elan Teixeira Alves\\
	Orientador(a): Prof. Mestre Leonardo Reis Lucena
\end{flushright}

\vspace{1cm}

\begin{center}
	\Large{\textsc{\textbf{Resumo}}}
\end{center}

\noindent Na atual realidade brasileira, ter domínio sobre alguma língua estrangeira, principalmente o inglês, deixou de ser um diferencial e passou a ser uma obrigatoriedade para aqueles que almejam avançar profissionalmente nas áreas tecnológicas, de pesquisa ou até mesmo pessoais. Alcançar essa habilidade possibilita que novas oportunidades sejam criadas, tais como viagens pelo mundo, melhores ofertas de trabalho, viabilizar intercâmbio estudantil, entre outras. Existem diversas formas, já consolidadas, para aprendizagem de uma segunda língua, todavia, as mais efetivas se mostram por vezes dispendiosos para a população de baixa renda. Alternativas gratuitas podem ser encontradas na \abrv[WEB -- World Wide Web]{WEB}, todavia não há um local unificado onde essas pessoas, que estão estudando algum idioma estrangeiro, possam interagir, exigindo que esses estudantes procurem outros com o mesmo interesse de aprendizagem em fóruns e em outras plataformas, necessitando que divulgue informações pessoais, como e-mail ou número, o que pode ser um problema, já que os dados estarão abertos para outros usuários. Neste cenário, foi idealizado um ambiente gratuito e aberto à comunidade para que os estudantes de novos idiomas consigam praticar uns com os outros, construindo uma comunidade que compartilha conhecimento. Para validar a ideia, foi realizado uma pesquisa através de formulários e conversas, com a finalidade de moldar uma maneira de solucionar este problema. Como resultado, um sistema web foi desenvolvido, com o mínimo de funcionalidades planejadas, com o intuito de ser consolidado nacionalmente e posteriormente ser difundido em outros locais no mundo. Abrangendo interação via vídeo e texto, sua fase de concepção foi realizada principalmente no ambiente do \abrv[IFRN -- Instituto Federal do Rio Grande do Norte]{IFRN}.

\noindent\textit{Palavras-chave}: Novo idiona, Estudar, Plataforma web, Comunidade.