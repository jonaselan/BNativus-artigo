% Introdu��o
\chapter{Introdução}

Este trabalho apresenta o desenvolvimento e desafios na construção do BNativus,  uma plataforma web que possibilita a estudantes compartilhar e aprender novos conhecimentos sobre idiomas, costumes e cultura. 
Nos dias de hoje, ter domínio sobre alguma língua estrangeira, principalmente o inglês, deixou de ser um diferencial e passou a ser uma obrigatoriedade para aqueles que possuem grandes objetivos profissionais nas áreas tecnológicas, de pesquisa ou até mesmo pessoais. Na perspectiva profissional, as pessoas enxergam uma realidade superior do que em relação ao Brasil, em outros países que se mostram mais desenvolvidos, com uma qualidade de vida melhor e com mais chances de trabalho. Segundo o site Trading Economics [1], que mantém uma série de estatísticas sobre os mais diversos países, o Brasil, nos últimos cinco anos, teve um aumento na taxa de desemprego inversamente proporcional à taxa de aumento de profissionais empregados. Em contrapartida, em países como Irlanda e Canadá a situação foi contrária.

 Na perspectiva pessoal, há o desejo do conhecimento, criando oportunidades de novas conexões internacionais, evitando o problema de comunicação em viagens ao exterior e em formações acadêmicas.

Esse tipo de experiência enriquece e amadurece as pessoas, que têm a possibilidade de conhecer novos costumes, pois aprender um idioma não se limita apenas a comunicação. Junto vem toda uma imersão cultural, que faz parte também daquela língua (costumes, gírias, expressões, história). E isso torna a conexão com pessoas, nativas ou não, mais profunda.

O fato é que, dominar um idioma estrangeiro possibilita novas oportunidades, por isso, cada vez mais são buscadas formas de aprender e aprimorar esse tipo de habilidade, seja escrita, leitura e/ou fala. Existem muitas maneiras de fazer isso. As formas mais consolidadas hoje são através de cursinhos e aulas particulares. Tecnologias aperfeiçoam esses métodos citados, inclusive, dando origem a novos, que pareciam impossíveis a alguns anos, como o ensino a distância. A WEB pode ser considerada como uma das maiores ferramentas, impactando a vida de todos (direta ou indiretamente).

O fácil acesso à internet permitiu mudar a forma como as pessoas se comunicam, se expressam e aprendem. O conteúdo encontrando nessa rede virtual mostra-se ilimitado, e com uma grande diversidade de conteúdo, facilitando o aprendizado autodidata de praticamente qualquer coisa. No entanto, é fato que nessa imensidão informacional, há também aquelas com informações de fontes não confiáveis. É, portanto,  essencial que os internautas fiquem atentos a isso.

No tocante ao estudo de uma nova língua, além de sites, existem inúmeras plataformas de ensino online que usam práticas já consolidadas de aprendizagem. Para exemplificar, podem ser citados o Duolingo [12] e Memrise [13] que usam a gamificação (mecânicas e dinâmicas de jogos usadas para, dentre outras coisas, melhorar o aprendizado). Há também o Quizlet [15] que utiliza flash cards (cartões nos quais são escritos uma palavra a ser aprendida e no verso é colocado a sua tradução para a língua mãe).

Essas aplicações são gratuitas, embora possuam planos pagos, nos quais o usuário tem total acesso a todas as funcionalidades do sistema. As aplicações já estão consolidadas no mercado para ensinar novas línguas, mas não se aplicam a todos os tipos de usuários. Aqueles que possuem nível um pouco acima do básico na conversação e escrita precisam de maneiras alternativas para dar continuidade aos estudos. Sendo assim, muitos procuram serviços de aulas online com professores estrangeiros, como o Verbling [10] e Italki [11].

Tais plataformas são estrangeiras e recebem o pagamento convertido em dólar. Com a cotação desta moeda atualmente, as aulas se tornam mais caras. A ausência de sistemas de conversas semelhantes a esses pode interferir no aprendizado daquelas que não possuem condições. Estudantes que estão matriculados em algum curso ainda possuem a possibilidade de realizar intercâmbio, uma experiência enriquecedora onde o estudante se encontra totalmente imerso em uma nova cultura enquanto estuda.

Nessas condições estudantes acabam procurando formas alternativas para suprir a necessidade de aprendizagem. Muitos vão em buscas de parceiros de estudo, uma prática comum onde, através de algum fórum ou grupo de aprendizado de línguas, pessoas disponibilizam seus contatos para que outros possam realizar contato e manter um diálogo constante, seja por mensagens, áudio ou até mesmo vídeo. No entanto, o problema visto nessa abordagem é que os dados do estudante acabam por ser divulgados na rede, e isso é um problema a aqueles que prezam por sua privacidade. Outro fator negativo é que esse trabalho é todo feito manualmente, ou seja, é preciso buscar alguém disponível para depois procurar algum veículo que torne possível a conversação.

Visto isso, o objeto de estudo proposto neste trabalho é a criação de uma plataforma para reunir essas pessoas, preenchendo, em um único sistema, a lacuna deixada pelo processo manual. Há diversas formas de abordar os estudos em uma segunda língua, logo seria necessário estabelecer as funcionalidades que seria integradas dentro do sistema e que assim melhor suprisse a necessidade dos usuários. Para isso foram feitas entrevistas com alguns professores de inglês e com professores da DIATINF, e dessa forma ter uma certa mentoria em relação tanto a funcionalidades quanto a abordagem aos usuários. 

Por fim, foi definido que os estudantes teriam a opção de criar grupos de conversa, e praticar a conversação da língua de estudo, em qualquer lugar do mundo. As salas criadas por um usuário ficariam visíveis para outros usuários, que por sua vez poderiam conectar-se e iniciar um diálogo. Para deixá-los mais confortáveis, os usuários informariam o nível de conhecimento em cada língua, para que possam entrar nas salas que correspondem aos seus interesses. Outro serviço disponível seriam a seção para criação de artigos e debates, dessa forma, qualquer um poderá contribuir trocando experiências e ideias. Com isso a prática da escrita e leitura seria realizada.


\section{Objetivos Gerais}

O objetivo geral deste trabalho é oferecer a estudantes, que não possuem condições para pagar um curso ou aulas particulares, uma forma gratuita de praticar os seus estudos em uma língua estrangeira.

\section{Objetivos Específicos}

O objetivo específico deste trabalho é a criação de uma plataforma web onde pessoas, em qualquer lugar do mundo, possam criar salas de conversação compartilhadas, dessa forma praticando sua compreensão e fala em outras línguas. Haveria também um espaço para escrever artigos e debater ideias, para, assim, exercitar a leitura e a escrita. Essas funcionalidades objetivam que os usuários consigam adquirir conhecimento de outras culturas, e principalmente, praticar com a comunidade os seus estudos em outra língua.

\section{Metodologia}

Na fase de concepção da plataforma, foi feita uma pesquisa em livros de ensino de línguas, diversas fontes digitais e em outras plataformas educacionais, para delimitar as funcionalidades e a abordagem que o sistema iria possuir. Após esse levantamento, e delimitando um escopo inicial, foram criados dois formulários, um para os brasileiros e o outro para estrangeiros. Com o feedback recebido foi possível refinar mais ainda o escopo. Então foi iniciada a criação de um protótipo, para disponibilizá-lo o quanto antes, possibilitando que que novas funcionalidades pudessem ser acrescentadas e bugs corrigidos.

\section{Organização do trabalho}

O trabalho é composto por mais três capítulos, além da introdução. O segundo capítulo prossegue destacando as principais ferramentas tecnológicas utilizadas durante o desenvolvimento do projeto, com uma breve descrição de cada uma. O terceiro capítulo aborda a construção do sistema de fato, expondo o que foi realizado em cada sprint e a metodologia que foi utilizada, com algumas capturas de telas. Por fim, no último capítulo, aborda-se a conclusão e o que foi alcançado com este projeto.
